% LaTeX Curriculum Vitae Template
%
% Copyright (C) 2004-2009 Jason Blevins <jrblevin@sdf.lonestar.org>
% http://jblevins.org/projects/cv-template/
%
% You may use use this document as a template to create your own CV
% and you may redistribute the source code freely. No attribution is
% required in any resulting documents. I do ask that you please leave
% this notice and the above URL in the source code if you choose to
% redistribute this file.
\documentclass[letterpaper]{article}

\usepackage{hyperref}
\usepackage{geometry}
\usepackage[UTF8]{ctex}
% Comment the following lines to use the default Computer Modern font
% instead of the Palatino font provided by the mathpazo package.
% Remove the 'osf' bit if you don't like the old style figures.
\usepackage[T1]{fontenc}
\usepackage[sc,osf]{mathpazo}

% Set your name here
\def\name{Gongheng Zhang}

% Replace this with a link to your CV if you like, or set it empty
% (as in \def\footerlink{}) to remove the link in the footer:
\def\footerlink{https://github.com/MakeBestForYou/GonghengZhang\_CV.git}

% The following metadata will show up in the PDF properties
\hypersetup{
  colorlinks = true,
  urlcolor = blue,
  pdfauthor = {\name},
  pdfkeywords = {geophysics, statistics, mathematics},
  pdftitle = {\name: Curriculum Vitae},
  pdfsubject = {Curriculum Vitae},
  pdfpagemode = UseOutlines
}

\geometry{
  body={6.5in, 8.5in},
  left=0.7in,
  top=1.25in
}

% Customize page headers
\pagestyle{myheadings}
\markright{\name}
\thispagestyle{empty}

% Custom section fonts
\usepackage{sectsty}
\sectionfont{\rmfamily\mdseries\Large}
\subsectionfont{\rmfamily\mdseries\itshape\large}

% Other possible font commands include:
% \ttfamily for teletype,
% \sffamily for sans serif,
% \bfseries for bold,
% \scshape for small caps,
% \normalsize, \large, \Large, \LARGE sizes.

% Don't indent paragraphs.
\setlength\parindent{0em}

% Make lists without bullets
\renewenvironment{itemize}{
  \begin{list}{}{
    \setlength{\leftmargin}{1.2em}
  }
}{
  \end{list}
}

\begin{document}

% Place name at left
\hspace{-0.3in}
{\huge \name}

% Alternatively, print name centered and bold:
%\centerline{\huge \bf \name}

\vspace{0.25in}

\begin{minipage}{0.55\linewidth}
  \href{https://www.sustech.edu.cn/}{Southern University of Science and Technology} \\ 
  \href{http://ess.sustc.edu.cn/}{Department of Earth and Space Science} \\
  Building 9\#, Chuangyuan\\
  1088 Xueyuan Avenue, Shenzhen 518055, P.R. China
\end{minipage}
\begin{minipage}{0.4\linewidth}
  \begin{tabular}{ll}
    Phone: & (+86) 15899869994 \\
    Fax: &  (0755) 88018700 \\
    Email: & \href{mailto:11930912@mail.sustech.edu.cn}{\tt 11930912@mail.sustech.edu.cn} \\
    Homepage: & \href{http://faculty.sustech.edu.cn/}{\tt http://faculty.sustech.edu.cn/} \\
  \end{tabular}
\end{minipage}


%\section*{Personal}
%
%\begin{itemize}
%\item Born on September 9, 1994.
%\item Gangzhou city, JiangXi, China.
%\end{itemize}


\section*{Education}
\vspace{-0.1in}
\begin{table}[htbp]
	\raggedright  % 显示位置为中间	
	% l代表左对齐,c代表居中,r代表右对齐 %字母的个数对应列数,|代表分割线
	\begin{tabular}{ l c c c }
		& & & \\[-6pt]  %可以避免文字偏上来调整文字与上边界的距离
		\quad B.S. & Geo-information science and technology & Ocean University of China & 2013-2017. \\  
		& & & \\[-6pt]  %可以避免文字偏上 
		\quad M.A.& Engineering mechanics & Harbin Institute of Technology & 2017-2019. \\
		& & & \\[-6pt]  %可以避免文字偏上
		\quad Ph.D.& Engineering mechanics & Southern University of Science and Technology & 2019-now.		
	\end{tabular}
\end{table}

%
%\begin{itemize}
%  \item B.S. Geo-information science and technology, Ocean University of China, 2013-2017.
%
%  \item M.A. Engineering mechanics, Harbin Institute of Technology, 2017-2019.
%
%  \item Ph.D. Engineering mechanics, Southern University of Science and Technology, 2019-now.
%\end{itemize}


%\section*{Employment}
%
%\begin{itemize}
%\item  china earthquake administration 2017.6--2017.9.
%
%\end{itemize}


\section*{Publications}

%\subsection*{Journal Articles}

\begin{itemize}
\item Ambient noise Rayleigh wave imaging in Sichaun Basin by high-modes dispersion curves.\\
\textbf{American Geophysical Union Fall meeting 2018}.(oral)
\item Study on the structure of Sichuan Basin by using high-order dispersion curve of background noise.\\
\textbf{Annual meeting of Chinese Geoscience Union 2018}.(poster)
\item Extrac dispersion curve from ambient noise based on frequency domain Bessel transform.\\
\textbf{Annual meeting of Chinese Geoscience Union 2019}.(oral)
\end{itemize}

%\subsection*{Proceedings}
%
%\begin{itemize}
%\item A generalized T-Test and measure of multivariate dispersion,
%  Proc. Second Berkeley Symposium of Mathematical Statistics and
%  Probability, 1951.
%\end{itemize}

\section*{Honors}
\vspace{-0.25in}
\begin{table}[htbp]
	\raggedright  % 显示位置为中间	
	% l代表左对齐,c代表居中,r代表右对齐 %字母的个数对应列数,|代表分割线
	\begin{tabular}{ l c c c }
		& & \\[-6pt]  %可以避免文字偏上来调整文字与上边界的距离
		\quad Outstanding academic dissertation & & Harbin Institute of Technology & \qquad 2019.6 \\
	    & & \\[-6pt]  %可以避免文字偏上来调整文字与上边界的距离
		\quad National Scholarship & ¥8000 & Harbin Institute of Technology & \qquad 2018.10 \\ 	 
		& & \\[-6pt]  %可以避免文字偏上来调整文字与上边界的距离
		\quad Outstanding Graduates & & Ocean University of China & \qquad 2017.6 \\  
		& & \\[-6pt]  %可以避免文字偏上 
		\quad National Encouragement Scholarship & ¥5000 & Ocean University of China & \qquad 2016.10 \\ 
		& & \\[-6pt]  %可以避免文字偏上
		\quad National Scholarship & ¥8000 & Ocean University of China & \qquad 2014.10 \\ 	
	\end{tabular}
\end{table}





\bigskip

% Footer
\begin{center}
  \begin{footnotesize}
%    Last updated: \today \\
    \href{\footerlink}{\texttt{\footerlink}}
  \end{footnotesize}
\end{center}

\end{document}